\documentclass[12pt]{article}
\usepackage{hyperref}

\usepackage{graphicx}

\usepackage{mathtools}
\DeclarePairedDelimiter\ceil{\lceil}{\rceil}
\DeclarePairedDelimiter\floor{\lfloor}{\rfloor}


\usepackage{amsmath}
\usepackage{amsfonts}
\usepackage{amssymb}

\usepackage{listings}
\lstset{language=C++,
                basicstyle=\ttfamily,
                keywordstyle=\color{blue}\ttfamily,
                stringstyle=\color{red}\ttfamily,
                commentstyle=\color{green}\ttfamily,
                morecomment=[l][\color{magenta}]{\#}
}

\usepackage{breqn}
\usepackage[]{algorithm2e}
\usepackage{physics}

\newcommand\fnurl[2]{%
  \href{#2}{#1}\footnote{\url{#2}}%
}

\newcommand{\Four}{\mathcal{F}}
\renewcommand{\(}{\left(}
\renewcommand{\)}{\right)}

\title{A brief rocFFT tutorial}



\begin{document}
\maketitle

\section{Introduction}

\texttt{rocFFT} is a software library for computing Fast Fourier
Transforms (FFT) written in HIP. It is part of AMD's software
ecosystem based on \href{https://github.com/RadeonOpenCompute}{ROCm}.

\texttt{hipFFT} is a wrapper for \texttt{rocFFT} and \texttt{cuFFT},
allowing one to use either library via a unified interface.

\section{Installation}

\texttt{rocFFT} runs on Linux, and requires a HIP compiler, such as
\texttt{hipcc} (available as part of the ROCm ecosystem) or
\texttt{hip-clang}.

Pre-built binaries can be installed via \texttt{.deb} or \texttt{.rpm}
packages, which can be found at
\href{https://rocm.github.io/install.html#installing-from-amd-rocm-repositories}{ROCm's
  package servers}.  One can also install from source; the lates
release is available one
\href{https://github.com/ROCmSoftwarePlatform/rocFFT/releases/latest}{github}.
After downloading the \texttt{tar.gz}, one runs
\begin{verbatim}
tar -xvf v*.tar.gz
cd rocFFT-*
mkdir build && cd build
cmake -DCMAKE_CXX_COMPILER=hcc -DCMAKE_INSTALL_PREFIX=~/rocfft ..
make && make install
\end{verbatim}
Afterwards, one adds the resulting directories to one's
\texttt{CMAKE_PREFIX_PATH} or \texttt{CPLUS_INCLUDE_PATH} and
\texttt{LD_LIBRARY_PATH} as required.

There is also an installation script, \texttt{install.sh}, which can
also take care of dependencies on supported Linux distributions.

\section{Example}

FIXME: in-place, out-of-place?

We are going to use hipfft examples here.

\subsection{1D complex}


\subsection{1D real}

\subsection{1D complex}


\subsection{1D real}




\end{document}
