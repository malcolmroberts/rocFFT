\documentclass[12pt]{article}
\usepackage{hyperref}

\usepackage{graphicx}

\usepackage{mathtools}
\DeclarePairedDelimiter\ceil{\lceil}{\rceil}
\DeclarePairedDelimiter\floor{\lfloor}{\rfloor}


\usepackage{amsmath}
\usepackage{amsfonts}
\usepackage{amssymb}

\usepackage{listings}
\lstset{language=C++,
captionpos=b,
%numbers=left,
%numberstyle=\tiny,
frame=single,
showspaces=false,
showtabs=false,
breaklines=true,
showstringspaces=false,
breakatwhitespace=true,
escapeinside={(*@}{@*)},
commentstyle=\color{greencomments},
morekeywords={partial, var, value, get, set},
keywordstyle=\color{bluekeywords},
stringstyle=\color{redstrings},
basicstyle=\ttfamily\small,
}


\usepackage{breqn}
\usepackage[]{algorithm2e}
\usepackage{physics}

\newcommand\fnurl[2]{%
  \href{#2}{#1}\footnote{\url{#2}}%
}

\newcommand{\Four}{\mathcal{F}}
\renewcommand{\(}{\left(}
\renewcommand{\)}{\right)}

\title{A brief rocFFT tutorial}

\date{}


\begin{document}
\maketitle

\section{Introduction}

\texttt{rocFFT} is a software library, written in HIP, for computing
Fast Fourier Transforms (FFT) on the GPU. It is part of AMD's software
ecosystem based on \href{https://github.com/RadeonOpenCompute}{ROCm}.

\section{Installation}

\texttt{rocFFT} runs on Linux, and requires a HIP compiler, such as
\texttt{hcc}.

Pre-built binaries can be installed via \texttt{.deb} or \texttt{.rpm}
packages, which can be found at
\href{https://rocm.github.io/install.html#installing-from-amd-rocm-repositories}{ROCm's
  package servers}\footnote{\url{https://rocm.github.io/install.html\#installing-from-amd-rocm-repositories}}.

One can also install from source; the latest release is available one
\href{https://github.com/ROCmSoftwarePlatform/rocFFT/releases/latest}{github}\footnote{\url{https://github.com/ROCmSoftwarePlatform/rocFFT/releases/latest}}.
After downloading the \texttt{tar.gz}, one runs
\begin{verbatim}
tar -xvf v*.tar.gz
cd rocFFT-*
mkdir build && cd build
cmake -DCMAKE_CXX_COMPILER=hcc -DCMAKE_INSTALL_PREFIX=~/rocfft ..
make && make install
\end{verbatim}
Afterwards, one adds the resulting directories to one's
\texttt{CMAKE_PREFIX_PATH} or \texttt{CPLUS_INCLUDE_PATH} and
\texttt{LD_LIBRARY_PATH} as required.

There is also an installation script, \texttt{install.sh}, which can
also take care of dependencies on supported Linux distributions.

\section{Usage}

The \texttt{hipFFT} interface to \texttt{rocFFT} provides an
\texttt{FFTW}-like interface, with a simple interface for 1D, 2D, and
3D FFTs, and an advanced interface for batched transforms which also
allows for discontiguous data.  In-place and out-of-place transforms
are supported.

In order to use the \texttt{hipFFT} interface one must:
\begin{enumerate}
\item Create a plan
\item Execute the plan
\item Destroy the plan
\end{enumerate}
Sample code is given in the \url{clients/samples/hipfft} subdirectory,
with specific examples discussed below.

\subsection{1D complex}
For a 1D complex in-place transform, allocates
\lstinline{hipfftDoubleComplex* x} on the device, initializes the data
on the device, and creates and executes the plan via:
\begin{lstlisting}
hipfftHandle plan = NULL;
hipfftResult rc   = HIPFFT_SUCCESS;
rc                = hipfftCreate(&plan);
assert(rc == HIPFFT_SUCCESS);
rc = hipfftPlan1d(&plan,      // plan handle
                  Nx,         // transform length
                  HIPFFT_Z2Z, // transform type 
                  1);         // batch size (deprecated)
assert(rc == HIPFFT_SUCCESS);
rc = hipfftExecZ2Z(plan,       // plan handle
                   x,          // input data
                   x,          // output data
                   direction); // direction of transform
assert(rc == HIPFFT_SUCCESS);
hipfftDestroy(plan);
\end{lstlisting}
If one wants to perform a transform on \lstinline{float}s, one
replaces \lstinline{HIPFFT_Z2Z} with \lstinline{HIPFFT_C2C},
\lstinline{hipfftExecZ2Z} with \lstinline{hipfftExecC2C}, and declares
\lstinline{x} to be a pointer to \lstinline{hipfftComplex}.  If the
input data and output data arguments to \lstinline{hipfftExecZ2Z} are
equal, then the transform is in-place; otherwise, it is out-of-place.
The direction of the transform is either \lstinline{HIPFFT_FORWARD=-1}
or \lstinline{HIPFFT_BACKWARD=1}.

\subsection{1D real}

For real-to-complex transforms, the transform type is
\lstinline{HIPFFT_D2Z}, and the plan is executed with
\lstinline{hipfftExecD2Z}, (\lstinline{HIPFFT_R2C} and
\lstinline{hipfftExecR2C} for single-precision).  The direction is
implied by the choice of execute function, with
\lstinline{hipfftExecD2Z} forwards and \lstinline{hipfftExecZ2D}
backwards.  The transform takes \lstinline{Nx} real values to
\lstinline{Nx/2+1} complex values.  As usual, in-place transforms need
buffers large enough for both the input and the output, which means
allocating \lstinline{2*(Nx/2+1)} real values.

\subsection{2D complex}
2D complex/complex transforms are created via
\begin{lstlisting}
hipfftPlan2d(&plan,      // plan handle
             Nx, Ny,     // transform lengths
             HIPFFT_Z2Z, // transform type 
             1);         // batch size (deprecated)
\end{lstlisting}
with the data in row-major order.  Executing the plan is the same as
for the 1D case.

\subsection{2D real}
2D real-to-complex transforms take
$\text{\lstinline{Nx}}\times{}\text{\lstinline{Ny}}$ real values and
produce $\text{\lstinline{Nx}}\times{}\text{\lstinline{Ny/2+1}}$
complex values.  For in-place transforms, the input data at index
$\(\text{\lstinline{i}},\text{\lstinline{j}}\)$ can be found at
position \lstinline{i*2*(Ny/2+1)+j}, ie the stride is
\lstinline{2*(Ny/2+1)}.  For out-of-place transforms, the input stride
is just \lstinline{Ny}.

\subsection{3D transforms}
3D transform plans are created with \lstinline{hipfftPlan3d}, which
takes 3 lengths.

The real-to-complex transform maps
$\text{\lstinline{Nx}}\times{}\text{\lstinline{Ny}}\times{}\text{\lstinline{Nz}}$
real values to
$\text{\lstinline{Nx}}\times{}\text{\lstinline{Ny}}\times{}\text{\lstinline{Nz/2+1}}$
complex values.  For in-place transforms, the input data at index
$\(\text{\lstinline{i}},\text{\lstinline{j}},\text{\lstinline{k}}\)$
can be found at position \lstinline{(i*Ny+j)*2*(Nz/2+1)+k}, For
out-of-place transforms the position is \lstinline{(i*Ny+j)*Nz+k}.

\subsection{Batched transforms}

The advanced interface allows for batched transforms, which is more
efficient than repeatedly calling the basic interface when one has
several small transforms to compute and a regular memory structure.
Examples may be found in the \url{clients/samples/hipfft}
subdirectory.


\end{document}

%%  LocalWords:  rocFFT FFT GPU AMD's ROCm hcc Pre amd rocm ROCm's gz
%%  LocalWords:  github CMAKE CPLUS LD hipFFT FFTW FFTs discontiguous
%%  LocalWords:  subdirectory hipfftDoubleComplex hipfftHandle rc Nx
%%  LocalWords:  hipfftResult hipfftCreate hipfftPlan hipfftExecZ Ny
%%  LocalWords:  hipfftDestroy hipfftExecC hipfftComplex hipfftExecD
%%  LocalWords:  hipfftExecR ie Nz
